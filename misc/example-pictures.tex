\usetikzlibrary{mindmap}

% Keys to support piece-wise uncovering of elements in TikZ pictures:
% \node[visible on=<2->](foo){Foo}
%
% Internally works by setting opacity=0 when invisible, which has the
% adavantage (compared to \node<2->(foo){Foo} that the node is always there, hence
% always consumes space that (foo) is always available.
%
% The actual command that implements the invisibility can be overriden
% by altering the style invisible. For instance \tikzsset{invisible/.style={opacity=0.2}}
% would dim the "invisible" parts. Alternatively, the color might be set to white, if the
% output driver does not support transparencies (e.g., PS)
%
% see http://tex.stackexchange.com/questions/55806/tikzpicture-in-beamer/55827#55827
\tikzset{
  invisible/.style={opacity=0},
  visible on/.style={alt=#1{}{invisible}},
  alt/.code args={<#1>#2#3}{%
    \alt<#1>{\pgfkeysalso{#2}}{\pgfkeysalso{#3}} % \pgfkeysalso doesn't change the path
  },
}

\begin{frame}<1-9>[c]
    \frametitle{Vector graphics example (using PGF/TikZ)}
    \framesubtitle{Cuboid in a 2 vanishing points perspective}

    \footnotetext{Source: http://www.texample.net/tikz/examples/cuboid}
    \begin{figure}
        \begin{center}
            % \resizebox{!}{0.75\textheight}{
            \adjustbox{height=0.75\textheight}{
                \tikzsetfigurename{mindmap-}
                \nocachePicture
                \begin{tikzpicture}
                    \path[mindmap,concept color=ercisred,text=white]
                    node[concept] {Computer Science}
                    [clockwise from=0]
                    % note that `sibling angle' can only be defined in
                    % `level 1 concept/.append style={}'
                    child[concept color=ercisred!75, visible on=<2->] {
                    node[concept] {practical}
                    [clockwise from=90]
                    child[visible on=<3->] { node[concept] {algorithms} }
                    child[visible on=<4->] { node[concept] {data structures} }
                    child[visible on=<5->] { node[concept] {pro\-gramming languages} }
                    child[visible on=<6->] { node[concept] {software engineer\-ing} }
                    }
                    child[concept color=ercisblue!75, visible on=<7->] {
                    node[concept] {applied}
                    [clockwise from=-30]
                    child { node[concept] {databases} }
                    child { node[concept] {WWW} }
                    }
                    child[concept color=ercisred!25, visible on=<8->] { node[concept] {technical} }
                    child[concept color=ercisblue!25, visible on=<9->] { node[concept] {theoretical} };
                \end{tikzpicture}
            }
            \caption{Cuboid in a 2 vanishing points perspective.}
        \end{center}
    \end{figure}
\end{frame}

% \begin{frame}
%     \frametitle{Vector graphics example (using PGF/TikZ)}
%     \framesubtitle{Seismic focal mechanism in 3D view}
%     % See http://www.texample.net/media/tikz/examples/TEX/seismic-focal-mechanism-in-3d-view.tex
%     % \footnotetext{Source: http://www.texample.net/tikz/examples/seismic-focal-mechanism-in-3d-view}
%     \begin{center}
%         \nocachePicture
%         \tikzsetnextfilename{seismic-focal-}
%         \resizebox{!}{0.75\textheight}{
%             %: Styles for XYZ-Coordinate Systems
%             %: isometric  South West : X , South East : Y , North : Z
%             \tikzset{isometricXYZ/.style={x={(-0.866cm,-0.5cm)}, y={(0.866cm,-0.5cm)}, z={(0cm,1cm)}}}

%             %: isometric South West : Z , South East : X , North : Y
%             \tikzset{isometricZXY/.style={x={(0.866cm,-0.5cm)}, y={(0cm,1cm)}, z={(-0.866cm,-0.5cm)}}}

%             %: isometric South West : Y , South East : Z , North : X
%             \tikzset{isometricYZX/.style={x={(0cm,1cm)}, y={(-0.866cm,-0.5cm)}, z={(0.866cm,-0.5cm)}}}

%             \begin{tikzpicture} [scale=4, isometricZXY, line join=round, opacity=.75, text opacity=1.0, >=latex, inner sep=0pt, outer sep=2pt]
%                 \def\h{5}

%                 \newcommand{\quadrant}[2]{
%                     \foreach \t in {#1} \foreach \f in {175,165,...,5}
%                         \draw [fill=#2]
%                               ({sin(\f - \h)*cos(\t - \h)}, {sin(\f - \h)*sin(\t - \h)}, {cos(\f - \h)})
%                            -- ({sin(\f - \h)*cos(\t + \h)}, {sin(\f - \h)*sin(\t + \h)}, {cos(\f - \h)})
%                            -- ({sin(\f + \h)*cos(\t + \h)}, {sin(\f + \h)*sin(\t + \h)}, {cos(\f + \h)})
%                            -- ({sin(\f + \h)*cos(\t - \h)}, {sin(\f + \h)*sin(\t - \h)}, {cos(\f + \h)})
%                            -- cycle;
%                 }

%                 %Quadrants
%                 \quadrant{220,230,...,300}{black}
%                 \quadrant{-60,-50,...,20}{white}
%                 \quadrant{30,40,...,120}{black}
%                 \quadrant{130,140,...,210}{none}

%                 %Movement arrows
%                 \foreach \t in {225,235,...,295}
%                     \foreach \f in {50,40,...,0}
%                         \draw [red, opacity=1.0, ->, thick]
%                             ({sin(\f - \h)*cos(\t - \h)}, {sin(\f - \h)*sin(\t - \h)}, {cos(\f - \h)})
%                             -- ({(1 + 0.2*cos(90 - \f))*sin(\f - \h)*cos(\t - \h)},
%                                 {(1 + 0.2*cos(90 - \f))*sin(\f - \h)*sin(\t - \h)},
%                                 {(1 + 0.2*cos(90 - \f))*cos(\f - \h)});

%                 \foreach \t in {125,135,...,205}
%                     \foreach \f in {110,100,...,0}
%                         \draw [black, ->, thick]
%                             ({(1 + 0.2*cos(90 - \f))*sin(\f - \h)*cos(\t - \h)},
%                              {(1 + 0.2*cos(90 - \f))*sin(\f - \h)*sin(\t - \h)},
%                              {(1 + 0.2*cos(90 - \f))*cos(\f - \h)})
%                             -- ({sin(\f - \h)*cos(\t - \h)},{sin(\f - \h)*sin(\t - \h)},{cos(\f - \h)});
%                 \foreach \t in {35,45,...,115}
%                     \foreach \f in {130,120,...,0}
%                         \draw [red, opacity=1.0 ,->, thick]
%                             ({sin(\f - \h)*cos(\t - \h)}, {sin(\f - \h)*sin(\t - \h)}, {cos(\f - \h)})
%                             -- ({(1 + 0.2*cos(90 - \f))*sin(\f - \h)*cos(\t - \h)},
%                                 {(1 + 0.2*cos(90 - \f))*sin(\f - \h)*sin(\t - \h)},
%                                 {(1 + 0.2*cos(90 - \f))*cos(\f - \h)});

%                 \foreach \t in {-55,-45,...,25}
%                     \foreach \f in {130,120,...,0}
%                         \draw [black, ->, thick]
%                             ({(1 + 0.2*cos(90 - \f))*sin(\f - \h)*cos(\t - \h)},
%                              {(1 + 0.2*cos(90 - \f))*sin(\f - \h)*sin(\t - \h)},
%                              {(1 + 0.2*cos(90 - \f))*cos(\f - \h)})
%                           -- ({sin(\f - \h)*cos(\t - \h)},{sin(\f - \h)*sin(\t - \h)},{cos(\f - \h)});

%                 %Annotations
%                 \path ({1.5*sin(100)*cos(75)}, {1.5*sin(100)*sin(75)}, {1.5*cos(100)}) node [right] {Compression};
%                 \path ({1.5*sin(70)*cos(-15)}, {1.5*sin(70)*sin(-15)}, {1.5*cos(70)})  node [right] {Dilatation};
%                 \path ({1.25*sin(50)*cos(165)},{1.25*sin(50)*sin(165)},{1.25*cos(50)}) node [left]  {Dilatation};
%                 \path ({1.25*sin(30)*cos(255)},{1.25*sin(30)*sin(255)},{1.25*cos(30)}) node [left]  {Compression};

%                 %P and T axis
%                 \begin{scope}[ultra thick]
%                     \draw[->] ({1.75*sin(90)*cos(75)}, {1.75*sin(90)*sin(75)}, {1.75*cos(90)})
%                         -- ({2*sin(90)*cos(75)},{2*sin(90)*sin(75)},{2*cos(90)}) node [above] {T-axis};
%                     \draw[->] ({1.75*sin(90)*cos(255)},{1.75*sin(90)*sin(255)},{1.75*cos(90)})
%                         -- ({2*sin(90)*cos(255)},{2*sin(90)*sin(255)},{2*cos(90)}) node [below] {T-axis};
%                     \draw[<-] ({1.5*sin(90)*cos(-15)}, {1.5*sin(90)*sin(-15)}, {1.5*cos(90)})
%                         -- ({1.75*sin(90)*cos(-15)},{1.75*sin(90)*sin(-15)},{1.75*cos(90)}) node [right] {P-axis};
%                     \draw[<-] ({1.5*sin(90)*cos(165)}, {1.5*sin(90)*sin(165)}, {1.5*cos(90)})
%                         -- ({1.75*sin(90)*cos(165)},{1.75*sin(90)*sin(165)},{1.75*cos(90)}) node [left] {P-axis};
%                 \end{scope}
%             \end{tikzpicture}
%         }
%     \end{center}
% \end{frame}