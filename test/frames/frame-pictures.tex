\begin{frame}[c]
    \frametitle{Raster graphics example}
    \framesubtitle{Administrative center of the University of Münster}

    \begin{figure}
        \begin{center}
            \pgfimage[height=40mm]{images/university-hq.jpg}
            \caption{Built by the Baroque architect Johann Conrad Schlaun in 1767–87.}
        \end{center}
    \end{figure}
\end{frame}

\begin{frame}[c]
    \frametitle{Vector graphics example (using PDF)}
    \framesubtitle{ERCIS logo}

    \begin{figure}
        \begin{center}
            \pgfimage[height=40mm]{images/ercis-logo.pdf}
            \caption{A very well known logo in Münster.}
        \end{center}
    \end{figure}
\end{frame}

\begin{frame}<1-3|handout:3>[c]
    \frametitle{Vector graphics example (using PGF/TikZ)}
    \framesubtitle{Cuboid in a 2 vanishing points perspective}

    \footnotetext{Source: http://www.texample.net/tikz/examples/cuboid}
    \begin{figure}
        \begin{center}
            % \resizebox{!}{0.75\textheight}{
            % \adjustbox{height=0.75\textheight}{
                \tikzsetfigurename{cuboid-}
                \begin{tikzpicture}
                    \useasboundingbox (-2cm, -2.2cm) rectangle (3cm, 1cm);

                    %% Vanishing points for perspective handling
                    \coordinate (P1) at (-7cm,1.5cm); % left vanishing point (To pick)
                    \coordinate (P2) at (8cm,1.5cm); % right vanishing point (To pick)

                    %% (A1) and (A2) defines the 2 central points of the cuboid
                    \coordinate (A1) at (0em,0cm); % central top point (To pick)
                    \coordinate (A2) at (0em,-2cm); % central bottom point (To pick)

                    %% (A3) to (A8) are computed given a unique parameter (or 2) .8
                    % You can vary .8 from 0 to 1 to change perspective on left side
                    \coordinate (A3) at ($(P1)!.8!(A2)$); % To pick for perspective
                    \coordinate (A4) at ($(P1)!.8!(A1)$);

                    % You can vary .8 from 0 to 1 to change perspective on right side
                    \coordinate (A7) at ($(P2)!.7!(A2)$);
                    \coordinate (A8) at ($(P2)!.7!(A1)$);

                    %% Automatically compute the last 2 points with intersections
                    \coordinate (A5) at
                      (intersection cs: first line={(A8) -- (P1)},
                                second line={(A4) -- (P2)});
                    \coordinate (A6) at
                      (intersection cs: first line={(A7) -- (P1)},
                                second line={(A3) -- (P2)});

                    %%% Depending of what you want to display, you can comment/edit
                    %%% the following lines

                    %% Possibly draw back faces
                    \only<3->{
                        \fill[ercisblue!90] (A2) -- (A3) -- (A6) -- (A7) -- cycle; % face 6
                        \node at (barycentric cs:A2=1,A3=1,A6=1,A7=1) {\tiny f6};

                        \fill[ercisblue!50] (A3) -- (A4) -- (A5) -- (A6) -- cycle; % face 3
                        \node at (barycentric cs:A3=1,A4=1,A5=1,A6=1) {\tiny f3};

                        \fill[ercisblue!30] (A5) -- (A6) -- (A7) -- (A8) -- cycle; % face 4
                        \node at (barycentric cs:A5=1,A6=1,A7=1,A8=1) {\tiny f4};
                    }

                    \draw[thick,dashed] (A5) -- (A6);
                    \draw[thick,dashed] (A3) -- (A6);
                    \draw[thick,dashed] (A7) -- (A6);

                    %% Possibly draw front faces

                    % \fill[orange] (A1) -- (A8) -- (A7) -- (A2) -- cycle; % face 1
                    % \node at (barycentric cs:A1=1,A8=1,A7=1,A2=1) {\tiny f1};
                    \only<3->{
                        \fill[ercisblue!50,opacity=0.2] (A1) -- (A2) -- (A3) -- (A4) -- cycle; % f2
                        \node at (barycentric cs:A1=1,A2=1,A3=1,A4=1) {\tiny f2};
                        \fill[ercisblue!90,opacity=0.2] (A1) -- (A4) -- (A5) -- (A8) -- cycle; % f5
                        \node at (barycentric cs:A1=1,A4=1,A5=1,A8=1) {\tiny f5};
                    }

                    %% Possibly draw front lines
                    \draw[thick] (A1) -- (A2);
                    \draw[thick] (A3) -- (A4);
                    \draw[thick] (A7) -- (A8);
                    \draw[thick] (A1) -- (A4);
                    \draw[thick] (A1) -- (A8);
                    \draw[thick] (A2) -- (A3);
                    \draw[thick] (A2) -- (A7);
                    \draw[thick] (A4) -- (A5);
                    \draw[thick] (A8) -- (A5);

                    % Possibly draw points
                    % (it can help you understand the cuboid structure)
                    \foreach \i in {1,2,...,8}
                    {
                      \only<2->{
                          \draw[fill=black] (A\i) circle (0.15em)
                            node[above right] {\tiny \i};
                      }
                    }
                    % \draw[fill=black] (P1) circle (0.1em) node[below] {\tiny p1};
                    % \draw[fill=black] (P2) circle (0.1em) node[below] {\tiny p2};
                \end{tikzpicture}
            % }
            \caption{Cuboid in a 2 vanishing points perspective.}
        \end{center}
    \end{figure}
\end{frame}
