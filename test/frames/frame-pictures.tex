\begin{frame}[c]
    \frametitle{Raster graphics example}
    \framesubtitle{Administrative center of the University of Münster}

    \begin{figure}
        \begin{center}
            \pgfimage[height=40mm]{images/university-hq.jpg}
            \caption{Built by the Baroque architect Johann Conrad Schlaun in 1767–87.}
        \end{center}
    \end{figure}
\end{frame}

\begin{frame}[c]
    \frametitle{Vector graphics example (using PDF)}
    \framesubtitle{ERCIS logo}

    \begin{figure}
        \begin{center}
            \pgfimage[height=40mm]{images/ercis-logo.pdf}
            \caption{A very well known logo in Münster.}
        \end{center}
    \end{figure}
\end{frame}

\begin{frame}<1-3|handout:3>[c]
    \frametitle{Vector graphics example (using PGF/TikZ)}
    \framesubtitle{Cuboid in a 2 vanishing points perspective}

    \footnotetext{Source: http://www.texample.net/tikz/examples/cuboid}
    \begin{figure}
        \begin{center}
            % \resizebox{!}{0.75\textheight}{
            % \adjustbox{height=0.75\textheight}{
                \tikzsetfigurename{cuboid-}
                \begin{tikzpicture}
                    \useasboundingbox (-2cm, -2.2cm) rectangle (3cm, 1cm);

                    %% Vanishing points for perspective handling
                    \coordinate (P1) at (-7cm,1.5cm); % left vanishing point (To pick)
                    \coordinate (P2) at (8cm,1.5cm); % right vanishing point (To pick)

                    %% (A1) and (A2) defines the 2 central points of the cuboid
                    \coordinate (A1) at (0em,0cm); % central top point (To pick)
                    \coordinate (A2) at (0em,-2cm); % central bottom point (To pick)

                    %% (A3) to (A8) are computed given a unique parameter (or 2) .8
                    % You can vary .8 from 0 to 1 to change perspective on left side
                    \coordinate (A3) at ($(P1)!.8!(A2)$); % To pick for perspective
                    \coordinate (A4) at ($(P1)!.8!(A1)$);

                    % You can vary .8 from 0 to 1 to change perspective on right side
                    \coordinate (A7) at ($(P2)!.7!(A2)$);
                    \coordinate (A8) at ($(P2)!.7!(A1)$);

                    %% Automatically compute the last 2 points with intersections
                    \coordinate (A5) at
                      (intersection cs: first line={(A8) -- (P1)},
                                second line={(A4) -- (P2)});
                    \coordinate (A6) at
                      (intersection cs: first line={(A7) -- (P1)},
                                second line={(A3) -- (P2)});

                    %%% Depending of what you want to display, you can comment/edit
                    %%% the following lines

                    %% Possibly draw back faces
                    \only<3->{
                        \fill[ercisblue!90] (A2) -- (A3) -- (A6) -- (A7) -- cycle; % face 6
                        \node at (barycentric cs:A2=1,A3=1,A6=1,A7=1) {\tiny f6};

                        \fill[ercisblue!50] (A3) -- (A4) -- (A5) -- (A6) -- cycle; % face 3
                        \node at (barycentric cs:A3=1,A4=1,A5=1,A6=1) {\tiny f3};

                        \fill[ercisblue!30] (A5) -- (A6) -- (A7) -- (A8) -- cycle; % face 4
                        \node at (barycentric cs:A5=1,A6=1,A7=1,A8=1) {\tiny f4};
                    }

                    \draw[thick,dashed] (A5) -- (A6);
                    \draw[thick,dashed] (A3) -- (A6);
                    \draw[thick,dashed] (A7) -- (A6);

                    %% Possibly draw front faces

                    % \fill[orange] (A1) -- (A8) -- (A7) -- (A2) -- cycle; % face 1
                    % \node at (barycentric cs:A1=1,A8=1,A7=1,A2=1) {\tiny f1};
                    \only<3->{
                        \fill[ercisblue!50,opacity=0.2] (A1) -- (A2) -- (A3) -- (A4) -- cycle; % f2
                        \node at (barycentric cs:A1=1,A2=1,A3=1,A4=1) {\tiny f2};
                        \fill[ercisblue!90,opacity=0.2] (A1) -- (A4) -- (A5) -- (A8) -- cycle; % f5
                        \node at (barycentric cs:A1=1,A4=1,A5=1,A8=1) {\tiny f5};
                    }

                    %% Possibly draw front lines
                    \draw[thick] (A1) -- (A2);
                    \draw[thick] (A3) -- (A4);
                    \draw[thick] (A7) -- (A8);
                    \draw[thick] (A1) -- (A4);
                    \draw[thick] (A1) -- (A8);
                    \draw[thick] (A2) -- (A3);
                    \draw[thick] (A2) -- (A7);
                    \draw[thick] (A4) -- (A5);
                    \draw[thick] (A8) -- (A5);

                    % Possibly draw points
                    % (it can help you understand the cuboid structure)
                    \foreach \i in {1,2,...,8}
                    {
                      \only<2->{
                          \draw[fill=black] (A\i) circle (0.15em)
                            node[above right] {\tiny \i};
                      }
                    }
                    % \draw[fill=black] (P1) circle (0.1em) node[below] {\tiny p1};
                    % \draw[fill=black] (P2) circle (0.1em) node[below] {\tiny p2};
                \end{tikzpicture}
            % }
            \caption{Cuboid in a 2 vanishing points perspective.}
        \end{center}
    \end{figure}
\end{frame}

% \begin{frame}
%     \frametitle{Vector graphics example (using PGF/TikZ)}
%     \framesubtitle{Seismic focal mechanism in 3D view}
%     % See http://www.texample.net/media/tikz/examples/TEX/seismic-focal-mechanism-in-3d-view.tex
%     % \footnotetext{Source: http://www.texample.net/tikz/examples/seismic-focal-mechanism-in-3d-view}
%     \begin{center}
%         \nocachePicture
%         \tikzsetnextfilename{seismic-focal-}
%         \resizebox{!}{0.75\textheight}{
%             %: Styles for XYZ-Coordinate Systems
%             %: isometric  South West : X , South East : Y , North : Z
%             \tikzset{isometricXYZ/.style={x={(-0.866cm,-0.5cm)}, y={(0.866cm,-0.5cm)}, z={(0cm,1cm)}}}

%             %: isometric South West : Z , South East : X , North : Y
%             \tikzset{isometricZXY/.style={x={(0.866cm,-0.5cm)}, y={(0cm,1cm)}, z={(-0.866cm,-0.5cm)}}}

%             %: isometric South West : Y , South East : Z , North : X
%             \tikzset{isometricYZX/.style={x={(0cm,1cm)}, y={(-0.866cm,-0.5cm)}, z={(0.866cm,-0.5cm)}}}

%             \begin{tikzpicture} [scale=4, isometricZXY, line join=round, opacity=.75, text opacity=1.0, >=latex, inner sep=0pt, outer sep=2pt]
%                 \def\h{5}

%                 \newcommand{\quadrant}[2]{
%                     \foreach \t in {#1} \foreach \f in {175,165,...,5}
%                         \draw [fill=#2]
%                               ({sin(\f - \h)*cos(\t - \h)}, {sin(\f - \h)*sin(\t - \h)}, {cos(\f - \h)})
%                            -- ({sin(\f - \h)*cos(\t + \h)}, {sin(\f - \h)*sin(\t + \h)}, {cos(\f - \h)})
%                            -- ({sin(\f + \h)*cos(\t + \h)}, {sin(\f + \h)*sin(\t + \h)}, {cos(\f + \h)})
%                            -- ({sin(\f + \h)*cos(\t - \h)}, {sin(\f + \h)*sin(\t - \h)}, {cos(\f + \h)})
%                            -- cycle;
%                 }

%                 %Quadrants
%                 \quadrant{220,230,...,300}{black}
%                 \quadrant{-60,-50,...,20}{white}
%                 \quadrant{30,40,...,120}{black}
%                 \quadrant{130,140,...,210}{none}

%                 %Movement arrows
%                 \foreach \t in {225,235,...,295}
%                     \foreach \f in {50,40,...,0}
%                         \draw [red, opacity=1.0, ->, thick]
%                             ({sin(\f - \h)*cos(\t - \h)}, {sin(\f - \h)*sin(\t - \h)}, {cos(\f - \h)})
%                             -- ({(1 + 0.2*cos(90 - \f))*sin(\f - \h)*cos(\t - \h)},
%                                 {(1 + 0.2*cos(90 - \f))*sin(\f - \h)*sin(\t - \h)},
%                                 {(1 + 0.2*cos(90 - \f))*cos(\f - \h)});

%                 \foreach \t in {125,135,...,205}
%                     \foreach \f in {110,100,...,0}
%                         \draw [black, ->, thick]
%                             ({(1 + 0.2*cos(90 - \f))*sin(\f - \h)*cos(\t - \h)},
%                              {(1 + 0.2*cos(90 - \f))*sin(\f - \h)*sin(\t - \h)},
%                              {(1 + 0.2*cos(90 - \f))*cos(\f - \h)})
%                             -- ({sin(\f - \h)*cos(\t - \h)},{sin(\f - \h)*sin(\t - \h)},{cos(\f - \h)});
%                 \foreach \t in {35,45,...,115}
%                     \foreach \f in {130,120,...,0}
%                         \draw [red, opacity=1.0 ,->, thick]
%                             ({sin(\f - \h)*cos(\t - \h)}, {sin(\f - \h)*sin(\t - \h)}, {cos(\f - \h)})
%                             -- ({(1 + 0.2*cos(90 - \f))*sin(\f - \h)*cos(\t - \h)},
%                                 {(1 + 0.2*cos(90 - \f))*sin(\f - \h)*sin(\t - \h)},
%                                 {(1 + 0.2*cos(90 - \f))*cos(\f - \h)});

%                 \foreach \t in {-55,-45,...,25}
%                     \foreach \f in {130,120,...,0}
%                         \draw [black, ->, thick]
%                             ({(1 + 0.2*cos(90 - \f))*sin(\f - \h)*cos(\t - \h)},
%                              {(1 + 0.2*cos(90 - \f))*sin(\f - \h)*sin(\t - \h)},
%                              {(1 + 0.2*cos(90 - \f))*cos(\f - \h)})
%                           -- ({sin(\f - \h)*cos(\t - \h)},{sin(\f - \h)*sin(\t - \h)},{cos(\f - \h)});

%                 %Annotations
%                 \path ({1.5*sin(100)*cos(75)}, {1.5*sin(100)*sin(75)}, {1.5*cos(100)}) node [right] {Compression};
%                 \path ({1.5*sin(70)*cos(-15)}, {1.5*sin(70)*sin(-15)}, {1.5*cos(70)})  node [right] {Dilatation};
%                 \path ({1.25*sin(50)*cos(165)},{1.25*sin(50)*sin(165)},{1.25*cos(50)}) node [left]  {Dilatation};
%                 \path ({1.25*sin(30)*cos(255)},{1.25*sin(30)*sin(255)},{1.25*cos(30)}) node [left]  {Compression};

%                 %P and T axis
%                 \begin{scope}[ultra thick]
%                     \draw[->] ({1.75*sin(90)*cos(75)}, {1.75*sin(90)*sin(75)}, {1.75*cos(90)})
%                         -- ({2*sin(90)*cos(75)},{2*sin(90)*sin(75)},{2*cos(90)}) node [above] {T-axis};
%                     \draw[->] ({1.75*sin(90)*cos(255)},{1.75*sin(90)*sin(255)},{1.75*cos(90)})
%                         -- ({2*sin(90)*cos(255)},{2*sin(90)*sin(255)},{2*cos(90)}) node [below] {T-axis};
%                     \draw[<-] ({1.5*sin(90)*cos(-15)}, {1.5*sin(90)*sin(-15)}, {1.5*cos(90)})
%                         -- ({1.75*sin(90)*cos(-15)},{1.75*sin(90)*sin(-15)},{1.75*cos(90)}) node [right] {P-axis};
%                     \draw[<-] ({1.5*sin(90)*cos(165)}, {1.5*sin(90)*sin(165)}, {1.5*cos(90)})
%                         -- ({1.75*sin(90)*cos(165)},{1.75*sin(90)*sin(165)},{1.75*cos(90)}) node [left] {P-axis};
%                 \end{scope}
%             \end{tikzpicture}
%         }
%     \end{center}
% \end{frame}