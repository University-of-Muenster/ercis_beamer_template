\begin{frame}[c]
    \frametitle{\en{Tables}\de{Tabellen}}
    \framesubtitle{\en{Supported display formats}\de{Unterstützte Anzeigeformate}}

    \tikzstyle{ratiobox}=[draw, rectangle, text centered, inner sep=0mm, outer sep= 0mm]

    \begin{table}[H]%
        \begin{center}%
            \renewcommand{\arraystretch}{1.25}
            \begin{tabular}{>{\centering} m{2cm} >{\centering} m{15mm} >{\centering} m{15mm} >{\centering} m{15mm}}
                \hline
                \textbf{\en{Aspect ratio}\de{Format}} & \textbf{\en{Option}\de{Option}} & \textbf{\en{Width}\de{Breite}} & \textbf{\en{Height}\de{Höhe}} \tabularnewline
                \hline\noalign{\vskip 6pt}
                \nocachePicture\tikz{\node [ratiobox, text width=15mm, minimum height=11.25mm] {4:3};} & - & 128 mm & 96 mm \tabularnewline
                \nocachePicture\tikz{\node [ratiobox, text width=15mm, minimum height=9.375mm] {16:10};} & wide10 & 128 mm & 80 mm \tabularnewline
                \nocachePicture\tikz{\node [ratiobox, text width=15mm, minimum height=8.4375mm] {16:9};} & wide9 & 128 mm & 72 mm \tabularnewline
                \hline
            \end{tabular}
            \caption{\en{Supported display formats}\de{Unterstützte Anzeigeformate}}%
            \renewcommand{\arraystretch}{1}
        \end{center}%
    \end{table}%
\end{frame}